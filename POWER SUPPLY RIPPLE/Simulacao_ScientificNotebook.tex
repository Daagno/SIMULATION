
\documentclass{article}
%%%%%%%%%%%%%%%%%%%%%%%%%%%%%%%%%%%%%%%%%%%%%%%%%%%%%%%%%%%%%%%%%%%%%%%%%%%%%%%%%%%%%%%%%%%%%%%%%%%%%%%%%%%%%%%%%%%%%%%%%%%%%%%%%%%%%%%%%%%%%%%%%%%%%%%%%%%%%%%%%%%%%%%%%%%%%%%%%%%%%%%%%%%%%%%%%%%%%%%%%%%%%%%%%%%%%%%%%%%%%%%%%%%%%%%%%%%%%%%%%%%%%%%%%%%%
%TCIDATA{OutputFilter=LATEX.DLL}
%TCIDATA{Version=5.50.0.2953}
%TCIDATA{<META NAME="SaveForMode" CONTENT="1">}
%TCIDATA{BibliographyScheme=Manual}
%TCIDATA{Created=Thursday, August 18, 2016 21:32:58}
%TCIDATA{LastRevised=Wednesday, September 14, 2016 23:09:54}
%TCIDATA{<META NAME="GraphicsSave" CONTENT="32">}
%TCIDATA{<META NAME="DocumentShell" CONTENT="General\Blank Document">}
%TCIDATA{CSTFile=Math with theorems suppressed.cst}
%TCIDATA{PageSetup=72,72,72,72,0}
%TCIDATA{ComputeDefs=
%$\vspace{1pt}R_{1}=15$
%1$R_{L}=10$
%$C=2$
%2$V_{i}=5$
%$V_{0}=3.\,\allowbreak 808\,5$
%$v\left( t\right) =\frac{V_{i}R_{L}}{R_{1}+R_{L}}\left( 1-e^{\frac{-t\left(
%R_{1}+R_{L}\right) }{CR_{1}R_{L}}}\right) +\left( V_{0}e^{-\frac{t\left(
%R_{1}+R_{L}\right) }{CR_{1}R_{L}}}\right) $
%$v_{on}\left( t\right) =\frac{V_{i}R_{L}}{R_{1}+R_{L}}\left( 1-e^{\frac{%
%-t\left( R_{1}+R_{L}\right) }{CR_{1}R_{L}}}\right) +\left( V_{0}e^{-\frac{%
%t\left( R_{1}+R_{L}\right) }{CR_{1}R_{L}}}\right) $
%1$v_{o}\left( t\right) =\frac{V_{i}R_{L}}{R_{1}+R_{L}}\left( 1-e^{\frac{%
%-t\left( R_{1}+R_{L}\right) }{CR_{1}R_{L}}}\right) +\left( V_{0}e^{-\frac{%
%t\left( R_{1}+R_{L}\right) }{CR_{1}R_{L}}}\right) $
%1$v_{f}\left( t\right) =\frac{V_{i}R_{R}}{R_{1}+R_{R}}\left( 1-e^{\frac{%
%-t\left( R_{1}+R_{R}\right) }{CR_{1}R_{R}}}\right) +\left( V_{0}e^{-\frac{%
%t\left( R_{1}+R_{R}\right) }{CR_{1}R_{R}}}\right) $
%1$R_{R}=100$
%$K_{1}\left( V_{i}\right) =\frac{V_{i}(R_{L})}{R_{1}+R_{L}}$
%1$V_{i}=5$
%2$R_{L}=10$
%$k_{1}=2$
%$c\left( T\right) =\frac{5\left( \frac{2}{5}+\left( \frac{20}{23}-\frac{2}{5}%
%\right) e^{-T\left( \frac{1}{12}\right) }-\frac{20}{23}e^{-T\left( \frac{1}{%
%12}+\frac{23}{600}\right) }\right) }{\left( 1-e^{-T\left( \frac{1}{12}+\frac{%
%23}{600}\right) }\right) }$
%}

%TCIDATA{AllPages=
%F=36,\PARA{038<p type="texpara" tag="Body Text" >\hfill \thepage}
%}


\newtheorem{theorem}{Theorem}
\newtheorem{acknowledgement}[theorem]{Acknowledgement}
\newtheorem{algorithm}[theorem]{Algorithm}
\newtheorem{axiom}[theorem]{Axiom}
\newtheorem{case}[theorem]{Case}
\newtheorem{claim}[theorem]{Claim}
\newtheorem{conclusion}[theorem]{Conclusion}
\newtheorem{condition}[theorem]{Condition}
\newtheorem{conjecture}[theorem]{Conjecture}
\newtheorem{corollary}[theorem]{Corollary}
\newtheorem{criterion}[theorem]{Criterion}
\newtheorem{definition}[theorem]{Definition}
\newtheorem{example}[theorem]{Example}
\newtheorem{exercise}[theorem]{Exercise}
\newtheorem{lemma}[theorem]{Lemma}
\newtheorem{notation}[theorem]{Notation}
\newtheorem{problem}[theorem]{Problem}
\newtheorem{proposition}[theorem]{Proposition}
\newtheorem{remark}[theorem]{Remark}
\newtheorem{solution}[theorem]{Solution}
\newtheorem{summary}[theorem]{Summary}
\newenvironment{proof}[1][Proof]{\noindent\textbf{#1.} }{\ \rule{0.5em}{0.5em}}
\input{tcilatex}
\begin{document}


\section{$\protect\vspace{1pt}$C\'{A}LCULO E SIMULA\c{C}\~{A}O DE RIPPLE}

Autor: Daniel Agnoletto

Objetivo: pesquisa pessoal

No circuito abaixo, uma fonte V$_{\text{in}}$ alimenta uma malha formada por

um resistor s\'{e}rie e uma carga R$_{\text{L}}$ paralela a um capacitor C.
Este circuito simula, de forma muito simplificada, o funcionamento de um
circuito digital, onde chaves internas (mosfets) est\~{a}o constantemente
ligando e desligando com um per\'{\i}odo T dado pelo clock do circuito (ou
por um m\'{u}ltiplo deste). Quando a chave S$_{\text{1}}$ est\'{a} aberta, a
carga R$_{\text{L}}$ possui um valor R$_{\text{Lmin}}$, quando fechada, a
carga \'{e} R$_{\text{Lmax}}$. O objetivo \'{e} calcular a varia\c{c}\~{a}o
na tens\~{a}o V$_{\text{out}}$ (ripple) dados os valores de V$_{\text{in}}$,
R$_{\text{1}}$, R$_{\text{Lmin}}$, R$_{\text{Lmax}}$, C e o per\'{\i}odo do
chaveamento T. Ap\'{o}s o perfil do ripple ser obtido, uma an\'{a}lise
espectral da corrente em R$_{\text{1}}$ ser\'{a} realizado para avaliar
quais frequ\^{e}ncias comp\~{o}e o sinal e estabelecer como os valores dos
componentes do circuito influenciam na distribui\c{c}\~{a}o espectral. A
partir disso, os valores de C e R$_{\text{1}}$ poder\~{a}o ser calculados
para que os picos de frequ\^{e}ncia sejam os menores poss\'{\i}veis. Os
resultados desta analise s\~{a}o importantes uma vez que podem ajudar na solu%
\c{c}\~{a}o de problemas relacionados a emiss\~{o}es eletromagn\'{e}ticas
(emc).

\begin{center}
$\vspace{1pt}\FRAME{itbpFU}{4.0698in}{2.4829in}{0in}{\Qcb{Figura 1: modelo
simplificado de circuito digital}}{}{ckt.bmp}{\special{language "Scientific
Word";type "GRAPHIC";maintain-aspect-ratio TRUE;display "USEDEF";valid_file
"F";width 4.0698in;height 2.4829in;depth 0in;original-width
8.4371in;original-height 5.1249in;cropleft "0";croptop "1";cropright
"1";cropbottom "0";filename 'CKT.bmp';file-properties "XNPEU";}}$
\end{center}

$\vspace{1pt}$

$\vspace{1pt}$O circuito da Figura 1 foi baseado no modelo representado na
Figura 2 abaixo. Podemos observar que a alimenta\c{c}\~{a}o do uC foi
realizada atrav\'{e}s de uma trilha de VCC (em vermelho) e um plano de GND
(em azul). Al\'{e}m disso, o circuito possui um capacitor C em paralelo com
os pinos de alimenta\c{c}\~{a}o do circuito integrado. A trilha de VCC
possui uma resist\^{e}ncia \^{o}hmica R$_{\text{1}}$ do cobre. Este modelo 
\'{e} incompleto, visto que a trilha tamb\'{e}m possui uma indut\^{a}ncia
associada. A t\'{\i}tulo de simplicidade, as indut\^{a}ncias n\~{a}o est\~{a}%
o sendo consideradas nesta primeira an\'{a}lise. Note tamb\'{e}m que n\~{a}o
est\~{a}o sendo consideradas imped\^{a}ncias no GND pois, por simplicidade,
admite-se que a imped\^{a}ncia no plano de terra \'{e} extremamente baixa
para ter efeito significativo.

\vspace{1pt}

\FRAME{dtbpFU}{4.1165in}{2.22in}{0pt}{\Qcb{Figura 2: modelo simples de
alimenta\c{c}\~{a}o de circuito digital}}{}{frt.bmp}{\special{language
"Scientific Word";type "GRAPHIC";maintain-aspect-ratio TRUE;display
"USEDEF";valid_file "F";width 4.1165in;height 2.22in;depth
0pt;original-width 4.0672in;original-height 2.1793in;cropleft "0";croptop
"1";cropright "1";cropbottom "0";filename 'frt.bmp';file-properties "XNPEU";}%
}

$\vspace{1pt}$

$\vspace{1pt}$Para encontrar a forma de onda da tens\~{a}o V$_{\text{out}}$
vamos retirar as equa\c{c}\~{o}es para a chave S$_{\text{1}}$ aberta e
fechada. Para isso, vamos derivar as equa\c{c}\~{o}es genericamente com R$_{%
\text{L}}$ e posteriormente substituir por R$_{\text{Lmin}}$ e R$_{\text{Lmax%
}}$. Trabalhando no dom\'{\i}nio da frequ\^{e}ncia e utilizando o teorema
dos n\'{o}s, temos a seguinte express\~{a}o para a corrente $I\left(
s\right) $ do resistor $R_{1}$:

\vspace{1pt}

Equa\c{c}\~{a}o 1) $\ \ \ \ \frac{V_{in}\left( s\right) -V_{out}\left(
s\right) }{R_{1}}=I\left( s\right) $

\vspace{1pt}

O somat\'{o}rio das correntes na carga R$_{\text{L}}$ e no capacitor C nos d%
\'{a} outra equa\c{c}\~{a}o:

\vspace{1pt}

Equa\c{c}\~{a}o 2) $\ \ \ \ \frac{V_{out}\left( s\right) -\left( \frac{V_{c0}%
}{s}\right) }{\frac{1}{j\varpi C}}+\frac{V_{out}\left( s\right) }{R_{L}}%
=I\left( s\right) $

\vspace{1pt}

Onde $V_{c0}$ \'{e} a tens\~{a}o inicial do capacitor.

\vspace{1pt}

$\left( V_{out}\left( s\right) -\left( \frac{V_{c0}}{s}\right) \right)
j\varpi C+\frac{V_{out}\left( s\right) }{R_{L}}=\frac{V_{in}\left( s\right)
-V_{out}\left( s\right) }{R_{1}}$

$\left( V_{out}\left( s\right) \right) j\varpi C-\left( \frac{V_{c0}}{s}%
\right) j\varpi C+\frac{V_{out}\left( s\right) }{R_{L}}+\frac{V_{out}\left(
s\right) }{R_{1}}=\frac{V_{in}\left( s\right) }{R_{1}}$

$\left( V_{out}\left( s\right) \right) j\varpi CR_{1}-\left( \frac{V_{c0}}{s}%
\right) j\varpi CR_{1}+\frac{R_{1}}{R_{L}}V_{out}\left( s\right)
+V_{out}\left( s\right) =V_{in}\left( s\right) $

\vspace{1pt}

Logo, a fun\c{c}\~{a}o de transfer\^{e}ncia do circuito \'{e}:

$V_{out}\left( s\right) =\frac{\left( V_{in}\left( s\right)
+V_{c0}CR_{1}\right) R_{L}}{R_{1}+R_{L}+sCR_{1}R_{L}}$

Para uma fonte $V_{in}$ constante:

$V_{in}\left( s\right) =\frac{V_{in}}{s}$

\vspace{1pt}

Neste caso a tens\~{a}o $V_{out}$ no dom\'{\i}nio da frequ\^{e}ncia \'{e}:

\vspace{1pt}

Equa\c{c}\~{a}o 3)$\qquad V_{out}\left( s\right) =\frac{\left( \frac{V_{in}}{%
s}+V_{c0}CR_{1}\right) R_{L}}{R_{1}+R_{L}+sCR_{1}R_{L}}$

\vspace{1pt}

A transformada inversa (fun\c{c}\~{a}o no dom\'{\i}nio do tempo):

Equa\c{c}\~{a}o 4)$\qquad V_{out}\left( t\right) =\frac{V_{in}R_{L}}{%
R_{1}+R_{L}}\left( 1-e^{\frac{-t\left( R_{1}+R_{L}\right) }{CR_{1}R_{L}}%
}\right) +V_{c0}e^{^{\frac{-t\left( R_{1}+R_{L}\right) }{CR_{1}R_{L}}}}$

\vspace{1pt}

Para S$_{\text{1}}$ fechada temos que $R_{L}=R_{L\min }$ e tens\~{a}o
inicial do capacitor $V_{c0}=V_{0}:$

$V_{out_{S1F}}\left( t\right) =\frac{V_{in}R_{L\min }}{R_{1}+R_{L\min }}%
\left( 1-e^{\frac{-t\left( R_{1}+R_{L\min }\right) }{CR_{1}R_{L\min }}%
}\right) +V_{0}e^{^{\frac{-t\left( R_{1}+R_{L\min }\right) }{CR_{1}R_{L\min }%
}}}$

Se utilizarmos as constantes $k_{1}=\frac{R_{L\min }}{R_{1}+R_{L\min }}$ e $%
\tau _{1}=\frac{\left( R_{1}+R_{L\min }\right) }{CR_{1}R_{L\min }}:$

Equa\c{c}\~{a}o 5)$\qquad V_{out_{S1F}}\left( t\right) =V_{in}k_{1}\left(
1-e^{-t\tau _{1}}\right) +V_{0}e^{-t\tau _{1}}$

\vspace{1pt}

De forma semelhante, para S$_{\text{1}}$ aberta temos:

$k_{2}=\frac{R_{L\max }}{R_{1}+R_{L\max }}$

$\tau _{2}=\frac{\left( R_{1}+R_{L\max }\right) }{CR_{1}R_{L\max }}$

\vspace{1pt}

\vspace{1pt}Equa\c{c}\~{a}o 6)$\qquad V_{out_{S1A}}\left( t\right)
=V_{in}k_{2}\left( 1-e^{-t\tau _{2}}\right) +V_{1}e^{-t\tau _{2}}$

\vspace{1pt}

Onde $V_{1}$ \'{e} a tens\~{a}o inicial do capacitor quando o ciclo com a
chave S$_{\text{1}}$ aberta \'{e} iniciado. Estamos admitindo que o tempo de
chave aberta \'{e} igual ao tempo de chave fechada $\frac{T}{2}$, ou seja,
uma onda quadrada. Como a chave est\'{a} sendo aberta e fechada com um per%
\'{\i}odo $T$, a tens\~{a}o inicial do capacitor no ciclo seguinte \'{e} a 
\'{u}ltima tens\~{a}o do capacitor no ciclo anterior. No regime transit\'{o}%
rio, as tens\~{o}es iniciais de cada ciclo mudam a cada per\'{\i}odo. No
regime estacion\'{a}rio, no entanto, as tens\~{o}es $V_{0}$ e $V_{1}$ s\~{a}%
o constantes pois o sistema j\'{a} atingiu o equil\'{\i}brio.

Assim, no regime estacion\'{a}rio, as tens\~{o}es iniciais $V_{0}$ e $V_{1%
\text{ }}$s\~{a}o dadas por:

\vspace{1pt}$V_{out_{S1F}}\left( \frac{T}{2}\right) =V_{1\text{ }}$ para o
intervalo de tempo $t=\frac{T}{2}$

$V_{out_{S1A}}\left( \frac{T}{2}\right) =V_{0\text{ }}$ para o intervalo de
tempo $t=\frac{T}{2}$

Substituindo esta rela\c{c}\~{a}o nas equa\c{c}\~{o}es 5 e 6 temos:

\vspace{1pt}

$V_{in}k_{1}\left( 1-e^{-\frac{T}{2}\tau _{1}}\right) +V_{0}e^{-\frac{T}{2}%
\tau _{1}}=V_{1\text{ }}$

$V_{in}k_{2}\left( 1-e^{-\frac{T}{2}\tau _{2}}\right) +V_{1}e^{-\frac{T}{2}%
\tau _{2}}=V_{0\text{ }}$

\vspace{1pt}

Lembrando que as rela\c{c}\~{a}o s\~{a}o v\'{a}lidas quando o sistema
estiver estacion\'{a}rio.

Reescrevendo as equa\c{c}\~{o}es acima temos:

Equa\c{c}\~{a}o 7)\qquad $V_{in}k_{1}+\left( V_{0}-V_{in}k_{1}\right) e^{-%
\frac{T}{2}\tau _{1}}=V_{1\text{ }}$

\qquad \qquad \qquad

Equa\c{c}\~{a}o 8)\qquad $V_{in}k_{2}+\left( V_{1}-V_{in}k_{2}\right) e^{-%
\frac{T}{2}\tau _{2}}=V_{0\text{ }}$

\vspace{1pt}

Substituindo as Equa\c{c}\~{o}es 8 em 7, temos as seguintes express\~{o}es
para $V_{1}$ e $V_{0}$:

\vspace{1pt}

Equa\c{c}\~{a}o 9)\qquad $V_{1}=\frac{V_{in}\left( k_{1}+\left(
k_{2}-k_{1}\right) e^{-\frac{T}{2}\tau _{1}}-k_{2}e^{-\frac{T}{2}\left( \tau
_{1}+\tau _{2}\right) }\right) }{\left( 1-e^{-\frac{T}{2}\left( \tau
_{1}+\tau _{2}\right) }\right) }$

\vspace{1pt}

Equa\c{c}\~{a}o 10)\qquad $V_{0}=\frac{V_{in}\left( k_{2}+\left(
k_{1}-k_{2}\right) e^{-\frac{T}{2}\tau _{2}}-k_{1}e^{-\frac{T}{2}\left( \tau
_{1}+\tau _{2}\right) }\right) }{\left( 1-e^{-\frac{T}{2}\left( \tau
_{1}+\tau _{2}\right) }\right) }$

\vspace{1pt}

O ripple de tens\~{a}o \'{e} dado por:

\vspace{1pt}

$V_{ripple}=V_{1}-V_{0}$

Equa\c{c}\~{a}o 11)$\qquad V_{ripple}=\frac{V_{in}\left( k_{1}-k_{2}\right)
\left( 1-e^{-\frac{T}{2}\tau _{1}}-e^{-\frac{T}{2}\tau _{2}}+e^{-\frac{T}{2}%
\left( \tau _{1}+\tau _{2}\right) }\right) }{\left( 1-e^{-\frac{T}{2}\left(
\tau _{1}+\tau _{2}\right) }\right) }$

\vspace{1pt}

Observe que se o decaimento for acentuado, podemos desconsiderar a
componente exponencial e o valor do ripple depender\'{a} t\~{a}o somente dos
valores finais de carga dos capacitores, como era de se esperar. Se o valor
de $T$ for grande o suficiente, ou os valores de capacit\^{a}ncia e resist%
\^{e}ncia forem demasiado pequenos, o capacitor vai carregar completamente
dentro do intervalo $\frac{T}{2}$ e o valor de ripple poder\'{a} ser
calculado diretamente dos valores finais de carga do capacitor para as situa%
\c{c}\~{o}es de $R_{L\min }$ e $R_{L\max }$.

\vspace{1pt}Exemplo:

Para $R_{1}=15\Omega $, $R_{L\min }=10\Omega $, $R_{L\max }=100\Omega $, $%
V_{in}=5V$, $C=100\mu F$, $T=1s$ temos:

$k_{1}=\frac{R_{L\min }}{R_{1}+R_{L\min }}=\frac{10}{15+10}=\allowbreak 
\frac{2}{5}$

$\tau _{1}=\frac{\left( R_{1}+R_{L\min }\right) }{CR_{1}R_{L\min }}=\frac{%
\left( 15+10\right) }{100\times 10^{-6}\left( 15\right) \left( 10\right) }%
=\allowbreak \frac{5000}{3}$

$k_{2}=\frac{R_{L\max }}{R_{1}+R_{L\max }}=\frac{100}{15+100}=\allowbreak 
\frac{20}{23}$

$\tau _{2}=\frac{\left( R_{1}+R_{L\max }\right) }{CR_{1}R_{L\max }}=\frac{%
\left( 15+100\right) }{100\times 10^{-6}\left( 15\right) \left( 100\right) }%
=\allowbreak \frac{2300}{3}$

$V_{0}=\frac{V_{in}\left( k_{2}+\left( k_{1}-k_{2}\right) e^{-\frac{T}{2}%
\tau _{2}}-k_{1}e^{-\frac{T}{2}\left( \tau _{1}+\tau _{2}\right) }\right) }{%
\left( 1-e^{-\frac{T}{2}\left( \tau _{1}+\tau _{2}\right) }\right) }=\frac{%
5\left( \frac{20}{23}+\left( \frac{2}{5}-\frac{20}{23}\right) e^{-\frac{1}{2}%
\left( \frac{2300}{3}\right) }-\frac{2}{5}e^{-\frac{1}{2}\left( \frac{2300}{3%
}+\frac{5000}{3}\right) }\right) }{\left( 1-e^{-\frac{1}{2}\left( \frac{2300%
}{3}+\frac{5000}{3}\right) }\right) }=\allowbreak 4.\allowbreak 347\,8$

$V_{1}=\frac{V_{in}\left( k_{1}+\left( k_{2}-k_{1}\right) e^{-\frac{T}{2}%
\tau _{1}}-k_{2}e^{-\frac{T}{2}\left( \tau _{1}+\tau _{2}\right) }\right) }{%
\left( 1-e^{-\frac{T}{2}\left( \tau _{1}+\tau _{2}\right) }\right) }=\frac{%
5\left( \frac{2}{5}+\left( \frac{20}{23}-\frac{2}{5}\right) e^{-\frac{1}{2}%
\left( \frac{5000}{3}\right) }-\frac{20}{23}e^{-\frac{1}{2}\left( \frac{2300%
}{3}+\frac{5000}{3}\right) }\right) }{\left( 1-e^{-\frac{1}{2}\left( \frac{%
2300}{3}+\frac{5000}{3}\right) }\right) }=\allowbreak 2.0$

$V_{out_{S1F}}\left( t\right) =V_{in}k_{1}+\left( V_{0}-V_{in}k_{1}\right)
e^{-t\tau _{1}}=5\left( \frac{2}{5}\right) +\left( \allowbreak
4.3478-5\left( \frac{2}{5}\right) \right) e^{-\frac{5000}{3}t}=$

$\allowbreak 2.\,\allowbreak 347\,8e^{-\frac{5000}{3}t}+2$

$V_{out_{S1A}}\left( t\right) =V_{in}k_{2}+\left( V_{1}-V_{in}k_{2}\right)
e^{-\frac{T}{2}\tau _{2}}=5\left( \frac{20}{23}\right) +\left( \allowbreak
2.0-5\left( \frac{20}{23}\right) \right) e^{-\frac{2300}{3}t}=\allowbreak $

$\frac{100}{23}-2.\,\allowbreak 347\,8e^{-\frac{2300}{3}t}\allowbreak $

\vspace{1pt}

\vspace{1pt}

\vspace{1pt}

\vspace{1pt}

\vspace{1pt}

\vspace{1pt}

\vspace{1pt}

\vspace{1pt}

\vspace{1pt}

\vspace{1pt}

\vspace{1pt}

\vspace{1pt}

\vspace{1pt}

\vspace{1pt}

\vspace{1pt}

\vspace{1pt}

\vspace{1pt}

\vspace{1pt}

\vspace{1pt}

$R_{1}=15$

$R_{L}=10$

$C=2$

$V_{i}=5$

$V_{0}=3.\,\allowbreak 808\,5$

$R_{R}=100$

$v_{o}\left( t\right) =\frac{V_{i}R_{L}}{R_{1}+R_{L}}\left( 1-e^{\frac{%
-t\left( R_{1}+R_{L}\right) }{CR_{1}R_{L}}}\right) +\left( V_{0}e^{-\frac{%
t\left( R_{1}+R_{L}\right) }{CR_{1}R_{L}}}\right) $

$v_{f}\left( t\right) =\frac{V_{i}R_{R}}{R_{1}+R_{R}}\left( 1-e^{\frac{%
-t\left( R_{1}+R_{R}\right) }{CR_{1}R_{R}}}\right) +\left( V_{0}e^{-\frac{%
t\left( R_{1}+R_{R}\right) }{CR_{1}R_{R}}}\right) $

$v_{o}(5)=\allowbreak 1.\,\allowbreak 808\,5e^{-\frac{5}{12}}+2$

$v_{f}\left( 50\right) \allowbreak =\allowbreak \frac{100}{23}%
-3.\,\allowbreak 666\,3e^{-\frac{23}{12}}\allowbreak =\allowbreak
3.\,\allowbreak 808\,5$

$v_{o}\left( 50\right) =\allowbreak 1.\,\allowbreak 808\,5e^{-\frac{25}{6}%
}+2=\allowbreak 2.\,\allowbreak 028$

$k_{1}=\frac{R_{L}}{R_{1}+R_{L}}=\frac{\left( 10\right) }{15+10}=\allowbreak 
\frac{2}{5}$

$k_{2}=\frac{R_{R}}{R_{1}+R_{R}}=\frac{100}{15+100}=\allowbreak \frac{20}{23}
$

$T_{1}=\frac{\left( R_{1}+R_{L}\right) }{CR_{1}R_{L}}=\frac{\left(
15+10\right) }{2\left( 15\right) 10}=\allowbreak \frac{1}{12}$

$T_{2}=\frac{\left( R_{1}+R_{R}\right) }{CR_{1}R_{R}}=\frac{\left(
15+100\right) }{2\left( 15\right) 100}=\allowbreak \frac{23}{600}$

$c\left( T\right) =\frac{5\left( \frac{2}{5}+\left( \frac{20}{23}-\frac{2}{5}%
\right) e^{-T\left( \frac{1}{12}\right) }-\frac{20}{23}e^{-T\left( \frac{1}{%
12}+\frac{23}{600}\right) }\right) }{\left( 1-e^{-T\left( \frac{1}{12}+\frac{%
23}{600}\right) }\right) }$

$c\left( 2\right) =\allowbreak -\frac{1}{e^{-\frac{73}{300}}-1}\left( \frac{%
54}{23}e^{-\frac{1}{6}}-\frac{100}{23}e^{-\frac{73}{300}}+2\right)
\allowbreak =V_{c1}\allowbreak =2.\,\allowbreak 679\,1$

$V_{c0}=5\left( \frac{20}{23}\right) +\left( 2.\,\allowbreak 679\,1-5\left( 
\frac{20}{23}\right) \right) e^{-2\left( \frac{23}{600}\right) }=\allowbreak
2.\,\allowbreak 802\,3$

$v_{o}\left( 2\right) =2\left( 1-e^{-\frac{1}{6}}\right) +\left( \allowbreak
2.8023e^{\frac{-1}{6}}\right) =\allowbreak 2.\,\allowbreak 679\,1$

$v_{f}\left( t\right) =\frac{\left( 5\right) \left( 100\right) }{15+100}%
\left( 1-e^{\frac{-2\left( 15+100\right) }{2\left( 15\right) \left(
100\right) }}\right) +\left( 2.6791e^{\frac{-2\left( 15+100\right) }{2\left(
15\right) \left( 100\right) }}\right) =\allowbreak 2.\,\allowbreak 802\,3$

\end{document}
